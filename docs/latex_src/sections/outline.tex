% outline.tex

\subsection{Outline}
This report outlines the steps taken to produce the system referred to as \textit{GNS (Geospatial Notebook System)}.\\
GNS aims to provide a field notebook experience for researchers by enabling location-tracking of their field notes. A \textit{raspberry pi} computer was used for the main functionality. A GPS device, and battery were connected to Pi. The GPS was used for fetching location data in the field, and the battery ensures operation of the device in the field.

\subsection{Setup and Configuration}
The setup of the system includes the following steps:

\begin{enumerate}
	\item Connect a GPS device/puck and install \textit{gpsd}
\item Configure GPSD with options: \begin{verbatim}GPSD_OPTIONS="-b -n"\end{verbatim} in the \textit{/etc/default/gpsd}
\item Setup the Wi-Fi hotspot by installing \textit{hostapd} and \textit{dnsmasq}
\item Configure a static IP by editing the configuration file: \textit{/etc/dhcpcd.conf}
\item Configure the access point by editing the \textit{/etc/hostapd/hostapd.conf} file. Ensure that the bridge is disabled.
\item Reboot the system and all services to access the wireless point.
\end{enumerate}

\noindent
Once the system is setup, development can take place. This would include the development of a backend system and a frontend web system. These systems are documented in preceeding sections of this report.

\subsection{Basic Workflow}
The basic work flow of GNS is given as follows:

\begin{enumerate}
	\item Start up the Raspberry Pi system, with the connected GPS, and battery.
	\item Connect to the Raspberry Pi's Wi-Fi hotspot: \textit{GNS-Hotspot}.
	\item Navigate to the GNS web service on any device near the Pi. This IP address is \textit{10.10.10.11}.
	\item A list of notes will be presented that are already in the system.
	\item Navigate using the web application to view, create, update, and delete notebooks as required.
\end{enumerate}

\subsection{Functionality}
The system was built to provide the following functionality to the user:\\

\begin{enumerate}
	\item \textbf{View Notes}\\
	The user would be allowed to login and view the notes that have been created so far. Each note would have in addition to the text information, geo-spatial data such as the latitude and longitude along with the data and time (UTC) of the note.
	
	\item \textbf{View Note}\\
	The user would be allowed to view specific details on a specific note. This includes its text information, the user who created the note, and the geo-spatial information.
	
	\item \textbf{Edit Note}\\
	The user would be allowed to make changes to a note. This includes editing the note's text data.
	
	\item \textbf{Add Note}\\
	The user would be allowed to add a new note to the system. The date and time and location will be fetched from the GPS device.
	
	\item \textbf{Delete Note}\\
	The user would be allowed to delete a note from the system. The note will no longer be visible to the users of the system.
	
\end{enumerate}
